\part{Using HPC}
\begin{frame}
\partpage
\end{frame}

\section{User Environment}
\begin{frame}{Using HPC: User Environment}
\begin{itemize}
\item<1,3->{\visible<1>{CentOS Linux 7.4 (}\alert<1>{{\color<3->{red}Red Hat Enterprise Linux 7}\visible<1>{.4 rebuild)}}}
\begin{itemize}
\item{\visible<1>{bash shell}}
\item{\visible<1>{Gnome or XFCE4 desktop \alert{(if you want)}}}
\item{\visible<1>{GCC compilers and other development software.}}
\end{itemize}
\item<2->{But you don't need to know that.}
\end{itemize}
\end{frame}

\subsection{Filesystems}
\begin{frame}{User Environment: Filesystems}
When you apply for an HPC account a home directory is created for you. 
\begin{itemize}
\item{\alert{/home/abc123}}
\begin{itemize}
\item{40GB quota.}
\item{Visible equally from all nodes.}
\item{Single storage server.}
\item{Hourly, daily, weekly snapshots copied to tape.}
\item{Not intended for job outputs or large/many input files.}
\end{itemize}
\item{\alert{/rds/user/abc123/hpc-work}}
\begin{itemize}
\item{Visible equally from all nodes.}
\item{Larger and faster (1TB initial quota).}
\item{Intended for job inputs and outputs.}
\item{{\color{red}Not backed up.}}
 \pause
\end{itemize}
\end{itemize}
\end{frame}

\subsection{Exceeding 1TB of HPC work}
\begin{frame}{Our storage services}
\begin{itemize}
\item{We have several storage services for users that need to exceed 1TB.}
\pause
\item{\alert{http://www.uis.cam.ac.uk/initiatives/storage-strategy/storage-services}}
\item{The most relevant services to HPC are RCS and RDS.}
\item{RCS - Research Cold Store is for data that isn't changing, data goes to disk then two sets of tapes.}
\item{RDS - Research Data Store, non backed up high performance storage for projects.}
\end{itemize}
\end{frame}

\subsection{Snapshots}
\begin{frame}{Filesystems: Snapshots}
\begin{itemize}
\item<1->{ZFS snapshots on /home are taken hourly, daily and weekly}
\item<2->{{\color{red}No backups are made of data in the RDS directories - take care when deleting.}}
\item<3->{Snapshots are not backups, they are retained for two weeks}
\item<4->{\color{red}Very short lived files are unlikely to get written to snapshots}
\item{The file needs exist long enough to get into a snapshot.}
\item{It is possible to search /home/.zfs/snapshot and browse the snapshots}
\item {Locatre the most recent version of the file then copy it back to your home folder} 
\end{itemize}
\end{frame}

\begin{frame}[fragile]{Filesystems: Quotas}
\begin{itemize}
\item{quota}
\begin{semiverbatim}
\tiny
[ps459@login-e-13 hpc-work]\$ quota
Filesystem  GiBytes    quota   limit   grace    files    quota    limit   grace User/group
/home          12.4     40.0    40.0       0    ----- No ZFS File Quotas  ----- U:ps459
/rds-d1        89.2   1024.0  1126.4       -   757345  1048576  1048576       - G:ps459
/rds-d1        22.1   1024.0  1024.0       -   113475  1048576  1048576       - G:rds-ps459-test
\end{semiverbatim}
\item<1-|handout:1->{\alert{Aim to stay below the soft limit (\emph{quota}).}}
\item<2-|handout:1->{\alert{Once over the soft limit, you have 7 days grace to return below.}}
\item<3-|handout:2>{\alert{When the grace period expires, or you reach the hard limit (\emph{limit}), no more data can be written.}}
\item<4-|handout:2>{\alert{It is important to rectify an out of quota condition ASAP.}}
\end{itemize}
\end{frame}

\begin{frame}{Filesystems: Permissions}
\begin{itemize}
\item{\color{red}Be careful and if unsure, please ask support.}
\begin{itemize}
\item{Can lead to \alert{accidental destruction} of your data or \alert{account compromise}.}
\end{itemize}
\item{Avoid changing the permissions on your home directory.}
\begin{itemize}
\item{Files under /home are particularly security sensitive.}
\item{Easy to break passwordless communication between nodes.}
\end{itemize}
\end{itemize}
\end{frame}

\section{Software}
\begin{frame}{Using HPC: Software}
\begin{itemize}
\item{Free software accompanying \alert{Red Hat Enterprise} is (or can be) provided.}
\item{Other software (free and non-free) is available via \alert{modules}.}
\item{Some proprietary software may not be generally accessible.}
\item{See \alert{http://www.hpc.cam.ac.uk/using-clusters/software}.}
\item{New software may be possible to provide on request.}
\item{\alert{Self-installed software must be properly licensed.}}
  \pause
\item{\color{red}\emph{sudo will not work.}\/ (You should be worried if it did.)}
\end{itemize}
\end{frame}

\subsection{Environment Modules}
\begin{frame}[fragile]{User Environment: Environment Modules}
\begin{itemize}
\item{Modules load or unload additional software packages.}
\item{Some are \alert{required} and automatically loaded on login.}
\item{Others are optional extras, or possible replacements for other modules.}
\item{\alert{Beware} unloading default modules in $\tilde{}\text{/.bashrc}$.}
\item{\alert{Beware} overwriting environment variables such as PATH and LD\_LIBRARY\_PATH in $\tilde{}\text{/.bashrc}$. If necessary append or prepend.}
\end{itemize}
\end{frame}

\subsection{Environment Modules}
\begin{frame}[fragile]{User Environment: Environment Modules}
\begin{itemize}
\item{Currently loaded:}
\begin{semiverbatim}
\scriptsize
module list
Currently Loaded Modulefiles:
 1) dot                            2) singularity/current            2) intel/impi/2017.4/intel       4) intel/libs/daal/2017.4
....plus several more default modules
\end{semiverbatim}
\medskip
\item{Available:}
\begin{semiverbatim}
\scriptsize
module av
\end{semiverbatim}
\end{itemize}
\end{frame}

\begin{frame}[fragile]{User Environment: Environment Modules}
\begin{itemize}
\item{Whatis:}
\begin{semiverbatim}
\tiny
module whatis openmpi-3.0.0-gcc-4.8.5-n2hvjgm
openmpi-3.0.0-gcc-4.8.5-n2hvjgm: The Open MPI Project is an open source...
\end{semiverbatim}
\medskip
\item{Load:}
\begin{semiverbatim}
\scriptsize
module load openmpi-3.0.0-gcc-5.4.0-4ryklvu
\end{semiverbatim}
\medskip
\item{Unload:}
\begin{semiverbatim}
\scriptsize
module unload openmpi-3.0.0-gcc-5.4.0-4ryklvu
\end{semiverbatim}
\end{itemize}
\end{frame}

\begin{frame}[fragile]{User Environment: Environment Modules}
\begin{itemize}
\item{Matlab}
\begin{semiverbatim}
\scriptsize
module load matlab/r2017b
\end{semiverbatim}
\medskip\pause
\item{Invoking matlab in batch mode:\hfill\\
  \qquad \alert{matlab -nodisplay -nojvm -nosplash command}\hfill\\
  where the file \alert{command.m} contains your matlab code.}
  \pause
  \item{The current site license contains the Parallel Computing Toolbox.}
\end{itemize}
\end{frame}

\begin{frame}[fragile]{User Environment: Environment Modules}
\begin{itemize}
\item{Purge and Load Defaults:}
\begin{semiverbatim}
\scriptsize
module purge
\item{module load rhel7/default-peta4}
\end{semiverbatim}
\smallskip
\end{itemize}
\end{frame}

\subsection{Compilers}
\begin{frame}[fragile]{User Environment: Compilers}
  \begin{itemize}
    \item{Load a newer GCC}
\begin{semiverbatim}
\scriptsize
module show gcc-7.2.0-gcc-4.8.5-pqn7o2k
module load gcc-7.2.0-gcc-4.8.5-pqn7o2k
Your commands to compile your software....
\end{semiverbatim}
\medskip
\item{If you have compiled software yourself your run time environment must match compile time environment!}

\begin{semiverbatim}
\scriptsize
gcc -O3 -mtune=native code.c -o prog
gfortran -O3 -mtune=native code.f90 -o prog
\medskip
module load openmpi-3.0.0-gcc-4.8.5-n2hvjgm
mpicc -O3 -mtune=native mpi_code.c -o mpi_prog
mpif90 -O3 -mtune=native mpi_code.f90 -o mpi_prog
\end{semiverbatim}
\pause
\end{itemize}
\end{frame}

\section{Excercises}

\subsection{Modules: Excercise 6}
\begin{frame}[fragile]{Excercise 6: Environment Modules}
\begin{itemize}
\item{Connect to the cluster using your training account: See excercise 1 if you have closed your terminal. }
\item{Get a list of modules that are currently loaded}
\item[\emph{Hints:}]{\alert{module list}}
\item{Get a list of available R modules}
\item[\emph{Hints:}]{\alert{module av R}}
\end{itemize}
\end{frame}

\subsection{Modules: Excercise 7}
\begin{frame}[fragile]{Excercise 7: Run an Rscript}
\begin{itemize}
\item{Connect to the cluster using your training account: See excercise 1 if you have closed your terminal.}
\item{In the exercises folder there is a file called test.r}
\item{Run this script using: Rscript hello.r }
\item{Load the module for: R/3.3.2}
\item[\emph{Hints:}]{\alert{module load R/3.3.2}}
\item{Run the script again: Rscript hello.r}
\end{itemize}
\end{frame}

\subsection{Local R library: Excercise 8}
\begin{frame}[fragile]{Exercise 8: Install the R library locally}
As a user you can create a local R library directory for packages that you want to install. 

\begin{itemize}
\item Load an R module: 
module load r-3.4.3-gcc-5.4.0-rbvhnga
\item Create a folder in your home for your own R package installs:
mkdir ~/my-R-libs
\item Make R aware of the new library location:
\begin{verbatim}
echo "R_LIBS_USER=~/my-R-libs" \textgreater .Renviron
\end{verbatim}
\item Start R:
R
\item Display your library paths:
.libPaths()
\item Try loading a library:
require(pander)
\item Its not insalled, lets install it:
install.packages("pander")
\item Try loading a library:
require(pander)
\item Library is now installed, lets quit R:
quit()
\end{itemize}
\end{frame}

\subsection{Excercise 8: notes}
\begin{frame}[fragile]{Excercise 8: Explained}
\begin{itemize}
\item Our R modules: module load r/(version). We have two sets of R modules, those with an upper case R where compiled for an older version of Linux and should be ignored (Darwin legacy)
\item echo " " outputs the text between the quotes, \textgreater redirects the text into the .Renviron file.
\item When we start R the .Renviron file is read and R will now be aware of our local library directory.
\item .libPaths() is how to check your library locations 
\end{itemize}
\end{frame}

\section{Excercise 9: Modules and Compilers}
\begin{frame}[fragile]{Exercise 9: Modules and Compilers}
\begin{itemize}
\item{Connect to the cluster using your training account: See excercise 1 if you have closed your terminal.}
\item{Go to the \alert{exercises} directory that you unzipped in hpcwork.}
\visible<1->{\begin{description}
\item{Try to compile the \alert{hello.c} program using the default \alert{gcc} compiler (it will fail because there is a deliberate bug).}
\end{description}}
\visible<2->{\begin{description}
\item[\emph{Hints:}]{\small \alert{gcc hello.c -o hello}}
\end{description}}
\item{To fix the problem, open the \alert{hello.c} file in the \alert{gedit} editor.}
\visible<3->{\begin{description}
\item[\emph{Hints:}]{\small Launch gedit in the background by doing \alert{gedit\&}. A gedit window should appear. Remove the word \alert{BUG}, save the file and recompile. Do \alert{./hello} to run the program.}
\end{description}}
\item{If you get this error: \begin{verbatim}WARNING **: cannot open display:\end{verbatim} then you have missed the '-Y' in your SSH command}
\end{itemize}
\end{frame}
 
\section{Excercise 9: Modules and Compilers}
\begin{frame}[fragile]{Exercise 9: Modules and Compilers}
\begin{itemize} 
\item{The default version of gcc is 4.8.5. Compile hello.c again with \alert{gcc 5.4.0}.}
\visible<1->{\begin{description}
\item[\emph{Hints:}]{\small module av, module load gcc-5.4.0-gcc-4.8.5-fis24gg, then \alert{gcc hello.c -o hello2}}
\end{description}}
\end{itemize}
\end{frame}

