\part{Section 4: Remote access to other Linux systems}
\begin{frame}
\partpage
\end{frame}

\section{Remote Access}
\begin{frame}{Section 4: Remote Linux systems}
\begin{itemize}
\item Most Linux systems allow remote log in
\item This is provided you have a user account on the remote machine
\item Most of the Linux systems you'll want to work with will be remote
\end{itemize}
\end{frame}

\section{Security}
\begin{frame}{Security}
\begin{enumerate}
\item{\alert{Keep your password (or private key passphrase) safe.}}
\pause
\item{\alert{Always choose strong passwords.}}
\pause
\item{\alert{Your UIS password is used for multiple systems so keep it secure!}}
\pause
\item{Keep the software on your laptops/tablets/PCs up to date this includes home computers especially if you are using the VPN to connect in.}
\pause
\item{Don't share accounts (this is against the rules anyway).}
\end{enumerate}
\end{frame}

\section{Remote Access Software}
\begin{frame}{Section 4: Remote Access Software}
\begin{itemize}
\item Remote access is provided by SSH
\item Files can be transferred by scp, sftp and rsync
\item There are other tools, we cover the ones that most systems have
\end{itemize}
\end{frame}

\section{Exercises}
\begin{frame}{Section 4: Exercises}
\begin{itemize}
\item In the notes go to Section 4: Remote Linux systems
\item {\textcolor{red}{We need to give you a username and password for the remote server}}
\item Read the notes for the section 
\item Attempt exercises 7 and 8
\item Raise your hand if you are stuck
\item We can demonstrate or explain an exercise
\end{itemize}
\end{frame}

\section{Command cheat sheet}
\begin{frame}{Section 2: Where am I?}
\begin{itemize}
\item At the back of your notes there is an SFTP cheat sheet
\end{itemize}
\end{frame}